\documentclass{article}
% pre\'ambulo

\usepackage{lmodern}
\usepackage[T1]{fontenc}
\usepackage[spanish,activeacute]{babel}
\usepackage{mathtools}
\usepackage{graphicx}
\usepackage{listings}
\usepackage{tabu}
\usepackage{hyperref}
\usepackage[utf8]{inputenc}
\usepackage{multicol}
\usepackage{amsmath}
\usepackage{amssymb}
\usepackage{enumerate}
\usepackage{amsthm}
\usepackage{wrapfig}
\usepackage{esvect}
\usepackage{subcaption}
\usepackage{wasysym}

\spanishdecimal{.}




% Default fixed font does not support bold face
\DeclareFixedFont{\ttb}{T1}{txtt}{bx}{n}{8} % for bold
\DeclareFixedFont{\ttm}{T1}{txtt}{m}{n}{8}  % for normal

% Custom colors
\usepackage[usenames,dvipsnames]{color}
\definecolor{deepblue}{rgb}{0,0,0.5}
\definecolor{deepred}{rgb}{0.6,0,0}
\definecolor{deepgreen}{rgb}{0,0.5,0}

% Python style for highlighting
\newcommand\pythonstyle{\lstset{
language=Python,
basicstyle=\ttm,
otherkeywords={self},             % Add keywords here
keywordstyle=\ttb\color{deepblue},
emph={MyClass,__init__},          % Custom highlighting
emphstyle=\ttb\color{deepred},    % Custom highlighting style
stringstyle=\color{deepgreen},
frame=tb,                         % Any extra options here
showstringspaces=false            % 
}}


% Python environment
\lstnewenvironment{python}[1][]
{
\pythonstyle
\lstset{#1}
}
{}

% Python for external files
\newcommand\pythonexternal[2][]{{
\pythonstyle
\lstinputlisting[#1]{#2}}}

% Python for inline
\newcommand\pythoninline[1]{{\pythonstyle\lstinline!#1!}}

\usepackage{amsmath} % or simply amstext
\newcommand{\angstrom}{\text{\normalfont\AA}}
\newcommand*{\everymodeprime}{\ensuremath{\prime}}

\title{Tarea 1}
\author{Francisco Felipe Carrasco Varela}

\usepackage{vmargin}

\setpapersize{A4}
\setmargins{1.82cm}       % margen izquierdo
{1.3cm}                        % margen superior
{17.5cm}                      % anchura del texto
{23.42cm}                    % altura del texto
{10pt}                           % altura de los encabezados
{1cm}                           % espacio entre el texto y los encabezados
{0pt}                             % altura del pie de página
{2cm}                           % espacio entre el texto y el pie de página

\usepackage{array,booktabs,tabularx,caption, ragged2e}
\newcolumntype{C}{>{\centering\arraybackslash}X}

\begin{document}
\begin{minipage}{2.3cm}
\includegraphics[width=2cm]{../logo_byn.png}
\vspace{0.5cm}
\end{minipage}
\begin{minipage}{\linewidth}
\textsc{\raggedright \footnotesize
Pontificia Universidad Católica de Chile \\
Facultad de Física -- Instituto de Astrof'isica \\
Astronom'ia -- AST0111 \\
Primer Semestre 2020}
\end{minipage}
\begin{center}
{\LARGE \textbf{Ayudant'ia 3}}

\vspace{3mm}

Profesora: Viviana Guzm'an

Ayudantes: Camila Aravena Gonz'alez (\texttt{cfaravena1$@$uc.cl}) -- Francisco Carrasco Varela (\texttt{ffcarrasco$@$uc.cl})

\end{center}
\begin{center}
\noindent\rule{12cm}{0.4pt}
\end{center}

Esta ayudant'ia tratar'a principalmente de repasar conceptos.

\vspace{3mm}

\textbf{Problema 1. D'ia sideral y d'ia solar}

\begin{enumerate} [a)]
\item ?`Cu'al es la diferencia entre el a'no tr'opico y el a'no sideral?

\item ?`A qu'e se debe esta diferencia?

\item ?`En cu'al de los dos a'nos se basa nuestro calendario? ?`Sideral o tr'opico?
\end{enumerate}

\vspace{5mm}

\textbf{Problema 2. Analema.} La Figura 1 muestra un analema. 

\begin{enumerate} [a)]

\item ?`Qu'e es un analema?

\item ?`Qu'e informaci'on podemos obtener de un analema si es que, adem'as, conocemos el hemisferio en el cual nos encontramos?

\item Dibuje en la Figura 1 en qu'e punto del analema el Sol se encuentra en el Solsticio de Verano, Invierno o el Ecuador Celeste. Para ello asume que el Sur se encuentra en la parte inferior del dibujo.
\end{enumerate}

\begin{figure}[!ht]
\begin{center}
\begin{tabular}{ll}
  \includegraphics[width=0.1\textwidth]{Analema.png}
\end{tabular}
\caption{Analema en alg'un lugar del hemisferio norte. Recuerde que en esta figura el Sur se encuentra en la parte inferior (o apuntando hacia abajo).}
\end{center} 
\end{figure}

\newpage

\textbf{Problema 3.} Suponga que usted observa todos los d'ias alg'un astro apenas este sale del horizonte. Si usted lo observ'o hoy, usted debe estimar a qu'e hora saldr'a mañana. 

?`Por qu'e este objeto va saliendo antes (o despu'es) del horizonte? Para ello le puede ser de utilidad realizar un dibujo.

\vspace{3mm}

\textbf{Problema 4.} Chile y su horario. Santiago, junto con todo Chile, deber'ia tener el horario \texttt{GMT $-5$}. Sin embargo, el que se usa es el horario \texttt{GMT $-4$}. Basados en esto, ?`a qu'e hora, aproximadamente, el Sol alcanza su m'axima altura y por qu'e?


\end{document}