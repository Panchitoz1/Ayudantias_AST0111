\documentclass{article}
% pre\'ambulo

\usepackage{lmodern}
\usepackage[T1]{fontenc}
\usepackage[spanish,activeacute]{babel}
\usepackage{mathtools}
\usepackage{graphicx}
\usepackage{listings}
\usepackage{tabu}
\usepackage{hyperref}
\usepackage[utf8]{inputenc}
\usepackage{multicol}
\usepackage{amsmath}
\usepackage{amssymb}
\usepackage{enumerate}
\usepackage{amsthm}
\usepackage{wrapfig}
\usepackage{esvect}
\usepackage{subcaption}
\usepackage{wasysym}

\spanishdecimal{.}




% Default fixed font does not support bold face
\DeclareFixedFont{\ttb}{T1}{txtt}{bx}{n}{8} % for bold
\DeclareFixedFont{\ttm}{T1}{txtt}{m}{n}{8}  % for normal

% Custom colors
\usepackage[usenames,dvipsnames]{color}
\definecolor{deepblue}{rgb}{0,0,0.5}
\definecolor{deepred}{rgb}{0.6,0,0}
\definecolor{deepgreen}{rgb}{0,0.5,0}

% Python style for highlighting
\newcommand\pythonstyle{\lstset{
language=Python,
basicstyle=\ttm,
otherkeywords={self},             % Add keywords here
keywordstyle=\ttb\color{deepblue},
emph={MyClass,__init__},          % Custom highlighting
emphstyle=\ttb\color{deepred},    % Custom highlighting style
stringstyle=\color{deepgreen},
frame=tb,                         % Any extra options here
showstringspaces=false            % 
}}


% Python environment
\lstnewenvironment{python}[1][]
{
\pythonstyle
\lstset{#1}
}
{}

% Python for external files
\newcommand\pythonexternal[2][]{{
\pythonstyle
\lstinputlisting[#1]{#2}}}

% Python for inline
\newcommand\pythoninline[1]{{\pythonstyle\lstinline!#1!}}

\usepackage{amsmath} % or simply amstext
\newcommand{\angstrom}{\text{\normalfont\AA}}
\newcommand*{\everymodeprime}{\ensuremath{\prime}}

\title{Tarea 1}
\author{Francisco Felipe Carrasco Varela}

\usepackage{vmargin}

\setpapersize{A4}
\setmargins{1.82cm}       % margen izquierdo
{1.3cm}                        % margen superior
{17.5cm}                      % anchura del texto
{23.42cm}                    % altura del texto
{10pt}                           % altura de los encabezados
{1cm}                           % espacio entre el texto y los encabezados
{0pt}                             % altura del pie de página
{2cm}                           % espacio entre el texto y el pie de página

\usepackage{array,booktabs,tabularx,caption, ragged2e}
\newcolumntype{C}{>{\centering\arraybackslash}X}

\begin{document}
\begin{minipage}{2.3cm}
\includegraphics[width=2cm]{../logo_byn.png}
\vspace{0.5cm}
\end{minipage}
\begin{minipage}{\linewidth}
\textsc{\raggedright \footnotesize
Pontificia Universidad Católica de Chile \\
Facultad de Física -- Instituto de Astrof'isica \\
Astronom'ia -- AST0111 \\
Primer Semestre 2020}
\end{minipage}
\begin{center}
{\LARGE \textbf{Ayudant'ia 1}}

\vspace{3mm}

Profesora: Viviana Guzm'an

Ayudantes: Camila Aravena Gonz'alez (\texttt{cfaravena1$@$uc.cl}) -- Francisco Carrasco Varela (\texttt{ffcarrasco$@$uc.cl})

\end{center}
\begin{center}
\noindent\rule{12cm}{0.4pt}
\end{center}


\textbf{Problema 1. Tamaño angular}

\vspace{3mm}

\begin{enumerate} [a)]

\item Si una pelota tiene un di'ametro de $70 \ cm$ y se encuentra a 35 metros de distancia. ?`Qu'e tamaño angular tiene? 

\item ?`A qu'e distancia debe estar si la quiero ver como un punto, digamos, $10$ veces m'as pequeño?

\item Pregunta para los curiosos: Por mera casualidad, el Sol y la Luna tienen una relaci'on asociada a su tamaño y distancia. ?`Qu'e nos permite ver esto (que ocurrir'a, por cierto, el 14 de Diciembre de este año)?

\end{enumerate}

\vspace{3mm}

\textbf{Problema 2. Paralaje}

\begin{enumerate} [a)]

\item ?`Qu'e es el paralaje y para qu'e se utiliza en astronom'ia? Diga, adem'as, sus ventajas y desventajas.
\item ?`Se le ocurre alguna manera ``casera'' de realizar paralaje en su casa?
\item El 'angulo de paralaje para la estrella Sirio es de $p = 1.05 \times 10^{-4} \ ^{\circ}$ (grados). ?`Qu'e tan lejos se encuentra esta estrella de nosotros? (Recuerde siempre tener cuidado con la conversión de unidades\footnote{$1 \ \text{arcsec} = 1^{\prime \prime} = \Big(\frac{1}{3600}\Big)^{\circ}$})
\item Dos personas con mucho tiempo de sobra se ponen de acuerdo para medir la distancia que hay entre la Tierra y Marte aprovechando que estos dos planetas se encontrar'an en oposici'on. Para ello, de aburridos, uno de ellos decide viajar a exactamente el otro lado del mundo (todo esto antes de la cuarentena, por supuesto; ociosos, pero responsables); de manera que la distancia entre los dos es igual al di'ametro de la Tierra\footnote{Radio de la Tierra $\equiv R_{\oplus} = 6.37 \times 10^{6} \ \text{m} \approx 6400 \ \text{km}$ }. El d'ia en que ambos planetas se encuentran en oposici'on, realizan sus mediciones, comparan sus datos de Marte y encuentran que el m'aximo cambio de posici'on angular entre sus mediciones es de $33.6^{\prime \prime}$, ?`cu'al es la distancia, aproximada, entre la Tierra y Marte cuando 'estos est'an en oposici'on? Diga sus respuestas en unidades de metros y AU\footnote{AU $\equiv$ Unidad Astron'omica (de sus siglas en ingl'es) $ = 1.49 \times 10^{11} \ \text{m} \approx 150 \times 10^{6} \ \text{km}$}.
\end{enumerate}

\vspace{4mm}

\textbf{Problema 3. Escalas de distancia}

\vspace{2mm}

\begin{enumerate} [a)]
\item Si una nave recorre la distancia que hay entre el Sol y la Tierra en dos años (obviamente, asumiendo que 'esta es pr'acticamente indestructible porque no se derrite, ni deja de funcionar al acercarse al Sol). ?`Cu'anto demorar'a esa misma nave en llegar a Marte, suponiendo que 'esta despega desde la Tierra? Para simplificar el problema, ya que la distancia entre la Tierra y Marte es variable (?`por qu'e?), asuma que los ingenieros hicieron los c'alculos de tal manera que la distancia que recorrer'a la nave es igual a la distancia a cuando la Tierra y Marte est'en en oposici'on; es decir, asuma que la distancia que recorrer'a la nave es igual a la respuesta que hall'o en el ejercicio 2.d).

\item ?`Cu'anto demorar'ian los tramos anteriores (Sol-Tierra y Tierra-Marte) si la nave pudiese viajar a la velocidad de la luz\footnote{Dato ñoño: M'as adelante en la carrera ver'a que es imposible que un objeto con masa viaje a esa velocidad}, es decir, a $3 \times 10^8 \ \text{m}\cdot \text{s}^{-1}$? 
\end{enumerate}

\vspace{4mm}




\end{document}