\documentclass{article}
% pre\'ambulo

\usepackage{lmodern}
\usepackage[T1]{fontenc}
\usepackage[spanish,activeacute]{babel}
\usepackage{mathtools}
\usepackage{graphicx}
\usepackage{listings}
\usepackage{tabu}
\usepackage{hyperref}
\usepackage[utf8]{inputenc}
\usepackage{multicol}
\usepackage{amsmath}
\usepackage{amssymb}
\usepackage{enumerate}
\usepackage{amsthm}
\usepackage{wrapfig}
\usepackage{esvect}
\usepackage{subcaption}
%\usepackage{wasysym}
\usepackage{mathabx}

\spanishdecimal{.}




% Default fixed font does not support bold face
\DeclareFixedFont{\ttb}{T1}{txtt}{bx}{n}{8} % for bold
\DeclareFixedFont{\ttm}{T1}{txtt}{m}{n}{8}  % for normal

% Custom colors
\usepackage[usenames,dvipsnames]{color}
\definecolor{deepblue}{rgb}{0,0,0.5}
\definecolor{deepred}{rgb}{0.6,0,0}
\definecolor{deepgreen}{rgb}{0,0.5,0}

% Python style for highlighting
\newcommand\pythonstyle{\lstset{
language=Python,
basicstyle=\ttm,
otherkeywords={self},             % Add keywords here
keywordstyle=\ttb\color{deepblue},
emph={MyClass,__init__},          % Custom highlighting
emphstyle=\ttb\color{deepred},    % Custom highlighting style
stringstyle=\color{deepgreen},
frame=tb,                         % Any extra options here
showstringspaces=false            % 
}}


% Python environment
\lstnewenvironment{python}[1][]
{
\pythonstyle
\lstset{#1}
}
{}

% Python for external files
\newcommand\pythonexternal[2][]{{
\pythonstyle
\lstinputlisting[#1]{#2}}}

% Python for inline
\newcommand\pythoninline[1]{{\pythonstyle\lstinline!#1!}}

\usepackage{amsmath} % or simply amstext
\newcommand{\angstrom}{\text{\normalfont\AA}}
\newcommand*{\everymodeprime}{\ensuremath{\prime}}

\title{Tarea 1}
\author{Francisco Felipe Carrasco Varela}

\usepackage{vmargin}

\setpapersize{A4}
\setmargins{1.82cm}       % margen izquierdo
{1.3cm}                        % margen superior
{17.5cm}                      % anchura del texto
{23.42cm}                    % altura del texto
{10pt}                           % altura de los encabezados
{1cm}                           % espacio entre el texto y los encabezados
{0pt}                             % altura del pie de página
{2cm}                           % espacio entre el texto y el pie de página

\usepackage{array,booktabs,tabularx,caption, ragged2e}
\newcolumntype{C}{>{\centering\arraybackslash}X}

\begin{document}
\begin{minipage}{2.3cm}
\includegraphics[width=2cm]{../logo_byn.png}
\vspace{0.5cm}
\end{minipage}
\begin{minipage}{\linewidth}
\textsc{\raggedright \footnotesize
Pontificia Universidad Católica de Chile \\
Facultad de Física -- Instituto de Astrof'isica \\
Astronom'ia -- AST0111 \\
Primer Semestre 2020}
\end{minipage}
\begin{center}
{\LARGE \textbf{Ayudant'ia 5}}

\vspace{3mm}

Profesora: Viviana Guzm'an

Ayudantes: Camila Aravena Gonz'alez (\texttt{cfaravena1$@$uc.cl}) -- Francisco Carrasco Varela (\texttt{ffcarrasco$@$uc.cl})

\end{center}
\begin{center}
\noindent\rule{12cm}{0.4pt}
\end{center}

\begin{center}
\begin{tabularx}{\linewidth}{*{6}{C}} \toprule
Planeta & Masa ($\rm kg$) & Volumen ($\rm km^3$) & Densidad media ($\rm \frac{gr}{cm^3}$) & {\small Per'iodo Rotaci'on} ({\small D'ias terrestres}) & Per'iodo Orbital ({\small A'nos terrestres}) \\\midrule
Mercurio ($\Mercury$) &  $3.3 \times 10^{23}$ & $6.08 \times 10^{10}$ & $5.42$ & $58.6$ & $0.24$\\
Venus ($\Venus$) & $4.86 \times 10^{24}$ & $9.28 \times 10^{11}$ & $5.24$ & $243$ & $0.61$\\
Tierra ($\Earth$) & $5.97 \times 10^{24}$ & $1.08 \times 10^{12}$ & $5.51$ & $1$ & $1.00$\\
Marte ($\Mars$) & $6.4 \times 10^{23}$ & $1.63 \times 10^{11}$& $3.93$& $1.03$ & $1.88$\\
J'upiter ($\Jupiter$) & $1.89 \times 10^{27}$ & $1.43 \times 10^{15}$ & $1.33$& $0.41$ & $11.86$\\
Saturno ($\Saturn$) & $5.68 \times 10^{26}$ & $8.27 \times 10^{14}$ & $0.68$ & $0.42$ & $29.46$\\
Urano ($\Uranus$) & $8.68 \times 10^{25}$ & $6.83 \times 10^{13}$ & $1.27$ & $0.71$ & $84.01$\\ 
Neptuno ($\Neptune$) & $1.02 \times 10^{26}$ & $6.25 \times 10^{13}$& $1.63$ & $0.67$ & $164.79$\\\bottomrule
\hline
\end{tabularx}
\captionof{table}{Algunos datos sobre los planetas del Sistema Solar.}  
\end{center}

\textbf{Problema 1. Caracter'isticas generales de planetas gaseosos (jovianos)}

\vspace{3mm}

\begin{enumerate}[a)] 

\item ¿Cu'ales son los planetas gaseosos?
\item ¿Cu'ales son los m'as grandes de ellos?
\item Observando el Cuadro 1, ¿qu'e puede decir si compara los planetas gaseosos con los rocosos?
\item ¿S'olo Saturno tiene ``anillos''? De ser negativa la respuesta, ¿cu'ales ser'ian los planetas?
\end{enumerate}

\vspace{3mm}

\textbf{Problema 2. Problemas generales}

\begin{enumerate}[a)]
\item Supongamos que observo dos estrellas y logro medir su color. Una de ellas resulta ser de color azul y la otra resulta ser de color rojo, ?`cu'al de las dos estrellas esperar'ia que fuera la m'as caliente y por qu'e?

\item ?`A qu'e distancia tendr'ia que estar una ampolleta de $100 \ W$ para que su flujo sea el mismo al flujo solar que recibe la Tierra? 

Para ello le puede servir de dato que la luminosidad del Sol es, aproximadamente, $L_\odot \approx 3.82 \times 10^{26} \ W \sim 4 \times 10^{26} \ W$.

\item El Hubble Space Telescope (HST) est'a a una distancia de $\sim 610$ km sobre la superficie de la Tierra, ubicada en una 'orbita aproximadamente circular alrededor de 'esta.




\begin{enumerate}[i)]
\item Estime su per'iodo orbital.

\item Los sat'elites de comunicaci'on y de monitoreos del clima usualmente est'an ubicados en lo que se conoce como 'orbitas de estacionamiento ``geos'incronas'' sobre la Tierra. La gracia de estas 'orbitas es que los sat'elites permanecen fijos sobre un punto espec'ifico sobre la Tierra. ?`A qu'e altitud deben estar localizados estos sat'elites?

\item ?`Es posible que un sat'elite en una 'orbita geos'incrona permanezca ``estacionado'' sobre cualquier punto sobre la Tierra. ?`Por qu'e o por qué no?
\end{enumerate}

%\item Las estrellas circumpolares son estrellas que nunca se ponen bajo el horizonte para el observador local; o estrellas que nunca son visibles sobre el horizonte (nunca salen). 

%\begin{enumerate}[i)]
%\item Calcule el rango de declinaciones para estos dos grupos de estrellas (que se ven y no se ven) para un observador a una latitud L.

%\item ?`A qu'e latitud(es) en la Tierra el Sol nunca se podr'a cuando sea el solsticio de verano (en el hemisferio norte)?

%\item Basado en los dos sub-problemas anteriores, ?`existe alguna latitud en la Tierra donde el Sol nunca se pondr'a cuando sea el equinnocio vernal? De ser as'i, ?`d'onde?
%\end{enumerate} 
\vspace{3mm}

\item Si nos encontrásemos en la ciudad de Lima (Perú), la cual tiene una latitud de $12^{\circ} S$ (o, equivalentemente, latitud de $-12^{\circ}$). 

\begin{enumerate} [i)]

\item ?`Cu'ales son las declinaciones (DEC) que se pueden ver en dicha ciudad?

\item ?`Cu'al es la m'axima altura sobre el horizonte que puede alcanzar una estrella cuyas coordenadas ser'an $RA = 3^h : 00^m : 00^s$ y $DEC = +20^{\circ}$?

\item ?`Cu'al es la mejor fecha para ver dicha estrella?

\item ?`Qu'e pasa si ahora observamos otro astro con la misma Ascensi'on Recta, pero con declinaci'on $DEC = +5^{\circ}$?

\item ?`Cu'al es entonces ``la mejor'' declinaci'on que puede tener una estrella para ser observada en las mejores condiciones posibles? Esto es, que pase justo a $90^{\circ}$ sobre el horizonte.

\item El centro gal'actico tiene coordenadas $RA = 17^h:45^m:40.04^s$ y $DEC = -29^{\circ} 00^{\prime} 28.1^{\prime \prime}$. ?`Qu'e puede decir sobre esto si lo relaciona con la posici'on de los telescopios gigantes que se encuentran en el Norte de Chile?
\end{enumerate}
\end{enumerate}


\end{document}