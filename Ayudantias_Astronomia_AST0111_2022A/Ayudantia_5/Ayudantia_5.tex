\documentclass{article}
% pre\'ambulo

\usepackage{lmodern}
\usepackage[T1]{fontenc}
\usepackage[spanish,activeacute]{babel}
\usepackage{mathtools}
\usepackage{graphicx}
\usepackage{listings}
\usepackage{tabu}
\usepackage{hyperref}
\usepackage[utf8]{inputenc}
\usepackage{multicol}
\usepackage{amsmath}
\usepackage{amssymb}
\usepackage{enumerate}
\usepackage{amsthm}
\usepackage{wrapfig}
\usepackage{esvect}
\usepackage{subcaption}
\usepackage{wasysym}

\spanishdecimal{.}




% Default fixed font does not support bold face
\DeclareFixedFont{\ttb}{T1}{txtt}{bx}{n}{8} % for bold
\DeclareFixedFont{\ttm}{T1}{txtt}{m}{n}{8}  % for normal

% Custom colors
\usepackage[usenames,dvipsnames]{color}
\definecolor{deepblue}{rgb}{0,0,0.5}
\definecolor{deepred}{rgb}{0.6,0,0}
\definecolor{deepgreen}{rgb}{0,0.5,0}

% Python style for highlighting
\newcommand\pythonstyle{\lstset{
language=Python,
basicstyle=\ttm,
otherkeywords={self},             % Add keywords here
keywordstyle=\ttb\color{deepblue},
emph={MyClass,__init__},          % Custom highlighting
emphstyle=\ttb\color{deepred},    % Custom highlighting style
stringstyle=\color{deepgreen},
frame=tb,                         % Any extra options here
showstringspaces=false            % 
}}


% Python environment
\lstnewenvironment{python}[1][]
{
\pythonstyle
\lstset{#1}
}
{}

% Python for external files
\newcommand\pythonexternal[2][]{{
\pythonstyle
\lstinputlisting[#1]{#2}}}

% Python for inline
\newcommand\pythoninline[1]{{\pythonstyle\lstinline!#1!}}

\usepackage{amsmath} % or simply amstext
\newcommand{\angstrom}{\text{\normalfont\AA}}
\newcommand*{\everymodeprime}{\ensuremath{\prime}}

\title{Ayudantia}
\author{Francisco Felipe Carrasco Varela}

\usepackage{vmargin}

\setpapersize{A4}
\setmargins{1.82cm}       % margen izquierdo
{1.3cm}                        % margen superior
{17.5cm}                      % anchura del texto
{23.42cm}                    % altura del texto
{10pt}                           % altura de los encabezados
{1cm}                           % espacio entre el texto y los encabezados
{0pt}                             % altura del pie de página
{2cm}                           % espacio entre el texto y el pie de página

\usepackage{array,booktabs,tabularx,caption, ragged2e}
\newcolumntype{C}{>{\centering\arraybackslash}X}

\begin{document}
\begin{minipage}{2.3cm}
\includegraphics[width=2cm]{../logo_byn.png}
\vspace{0.5cm}
\end{minipage}
\begin{minipage}{\linewidth}
\textsc{\raggedright \footnotesize
Pontificia Universidad Católica de Chile \\
Facultad de Física -- Instituto de Astrof'isica \\
Astronom'ia -- AST0111 \\
Primer Semestre 2022}
\end{minipage}
\begin{center}
{\LARGE \textbf{Ayudant'ia 5}}

\vspace{3mm}

Profesor: Mat'ias Bla'na D'iaz

Ayudante: Francisco Carrasco Varela (\texttt{ffcarrasco$@$uc.cl})

\end{center}
\begin{center}
\noindent\rule{12cm}{0.4pt}
\end{center}

\textbf{Problema 1. Efecto Doppler} (pendiente de la ayudantía pasada)

\begin{enumerate} [a)]

\item ¿Qué es el efecto Doppler? ¿Puede pensar en un ejemplo diario donde ocurra?

\item ¿Por qué el efecto Doppler sería util en Astronomía? ¿Tiene algún uso real/práctico?

\item ¿Qué son las líneas de absorción y cómo se pueden combinar con el efecto Doppler para conocer alguna característica de una estrella?

\item Suponga que usted observa el espectro de una estrella y mide que la línea de $H \alpha$ (léase como ``hache alfa'') se encuentra en la longitud de onda $6562 \ \AA$\footnote{$\AA = 10^{-10} \ \text{m}$} cuando en realidad ésta al ser medida en un laboratorio está a $6563 \ \AA$. ¿Cuál es la velocidad radial de la estrella? ¿Se está acercando o alejando de nosotros la estrella? 

\textcolor{red}{(R: $-45.68 \ \text{km} / \text{s}$)}

\item Una persona se pasa un semáforo en luz roja. Justo en ese lugar iba pasando una patrulla la cual ve a la persona pasándose la luz roja y la hace detenerse. Al hablar, el policía le indica claramente que se había pasado una luz roja y que, por tanto, debía cursarle una infracción. La persona afectada indica que esto no es así; que, en su defensa, él veía la luz de semáforo en verde. Luego de un rato discutiendo el policía decide darle la razón: la persona vió la luz del semáforo en verde, ¿se salva la persona de la infracción? ¿Por qué sí/no?
\end{enumerate}

\textbf{Problema 2. La luz}

\begin{enumerate} [a)]

\item ¿Qué es más energético: un fotón azul o uno rojo? ¿Por qué?

\item Supongamos que tenemos una ampolleta ultra eficiente (toda su energía se usa en luz y nada se pierde en calor) de 100 Watts la cual emite \emph{toda} su energía en verde. Si los fotones en verde tienen una frecuencia de $\sim 550 \ \text{nm}$\footnote{$1 \ \text{nm} = 10^{-9} \ \text{m}$}, ¿cuántos fotones de este color está emitiendo la ampolleta cada segundo, aproximadamente? Puede serle de utilidad saber que la constante de Planck es $h = 6.62 \times 10^{-34} \ \text{J} \cdot \text{s}$. 
\textcolor{red}{(R: Unos $2.77 \times 10^{20}$ fotones)}

\item ¿Qué son los conceptos de ``flujo'' y ``luminosidad'' y por qué son tan importantes en Astronomía? ¿Cómo se relacionan éstos entre sí?

\item Venus se encuentra a $0.49 \ \text{AU}$, ¿cómo se compara el flujo que recibe con el flujo que recibe de la Tierra? Más que responder con un número fijo al flujo en sí, simplemente indique cuántas veces es menor (o mayor) el flujo solar que recibe Venus con respecto al de la Tierra; no se complique la vida.

\newpage

\item Usted es un astrónomo que acaba de encontrar un nuevo sistema estrella-planetas (un símil al Sol y sus planetas, pero en un lugar muy, muy lejano) en el cual usted encontró dos planetas girando alrededor de esta estrella. Usted encuentra en su investigación que el planeta más cercano a la estrella estaba a $2 \ \text{AU}$ de distancia, mientras que el segundo se encontraba a $6 \ \text{AU}$ de distancia de su estrella. ¿Cómo se comparan los flujos que reciben estos planetas?

\item Se estima que en el mundo se consumen unos $160 \ 000 \ \text{TWh}$ (``Terawatt hora''), lo cual son aproximadamente unos $\sim 6 \times 10^{20} \ \text{W}$. El Sol tiene una luminosidad de $L_\odot = 3.82 \times 10^{26} \text{W}$. Si de alguna manera pudiésemos aprovechar toda la energía que emite el Sol por tan solo \emph{un} segundo, ¿por cuánto tiempo podría sustentar la energía solar obtenida el consumo global energético? \textcolor{red}{(R: El tiempo que la energía emitida por el Sol en 1 segundo podría sustentar toda la energía de la Tierra por un tiempo total de 7.3 días, aproximadamente)}
\end{enumerate}

\newpage

\begin{figure}
\centering
\begin{subfigure}[b]{0.75\textwidth}
   \includegraphics[width=1.1\linewidth]{vel_rad_1.jpg}
   \caption{}
   \label{fig:rad1} 
\end{subfigure}

\begin{subfigure}[b]{0.85\textwidth}
   \includegraphics[width=1\linewidth]{vel_rad_2.png}
   \caption{}
   \label{fig:rad2}
\end{subfigure}

\caption[Velocidades radiales]{Método de velocidades radiales para detección de planetas. Éste método, junto con el de detección/imagen directa y método de transición, son los métodos más usados para detectar planetas.}
\end{figure}


\end{document}